\documentclass[12pt, a4paper]{book}
\usepackage[utf8]{inputenc}
\usepackage[indonesian]{babel}
\usepackage{geometry}
\geometry{left=4cm, right=3cm, top=3cm, bottom=3cm}
\usepackage{setspace}
\doublespacing
\usepackage{graphicx}
\usepackage{natbib}
\bibliographystyle{apalike}
\renewcommand{\thechapter}{\Roman{chapter}}

\begin{document}

% Front Matter
\frontmatter

\begin{titlepage}
    \centering
    \vspace*{2cm}
    {\LARGE Usulan Penelitian}\\[1cm]
    {\Huge \textbf{JUDUL PENELITIAN}}\\[2cm]
    \includegraphics[width=0.5\textwidth]{logo.png}\\[1cm]
    {\Large Nama Mahasiswa}\\[0.5cm]
    {\large NIM: xxxxxxxxxx}\\[2cm]
    {\Large Program Studi XXXXX}\\[0.5cm]
    {\Large Pascasarjana}\\[0.5cm]
    {\Large Universitas Udayana}\\[1cm]
    {\Large Tahun}
\end{titlepage}

\chapter*{Persetujuan Pembimbing}
% Tambahkan nama pembimbing dan tanda tangan di sini.

\tableofcontents
\listoftables
\listoffigures
\chapter*{Daftar Singkatan dan Lambang}
\begin{description}
    \item[Contoh] Penjelasan
\end{description}

% Main Body
\mainmatter

\chapter{Pendahuluan}
\section{Latar Belakang}
% Tulis latar belakang penelitian Anda di sini.

\chapter{Kajian Pustaka}
% Jelaskan tinjauan pustaka yang relevan.

\chapter{Kerangka Berpikir, Konsep Penelitian dan Hipotesis Penelitian}
\section{Kerangka Berpikir}
% Deskripsikan kerangka berpikir.

\chapter{Metode Penelitian}
\section{Rancangan Penelitian}
% Jelaskan desain penelitian.

% Back Matter
\backmatter

\chapter*{Daftar Pustaka}
\bibliography{references}

\chapter*{Lampiran}
% Sertakan lampiran seperti jadwal penelitian.

\end{document}